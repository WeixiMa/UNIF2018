\documentclass[pdf]{beamer}
\mode<presentation>{}

\usepackage{textpos}
\usepackage{natbib}
\usepackage{proof}
 \usepackage[utf8]{inputenc}
 \usepackage{caption}
 %\usepackage[absolute,overlay]{textpos}
 \setbeamertemplate{footline}[frame number]
 \date{\today} 
 \subject{Thesis}  
 \usepackage{graphbox}
 \usepackage{tikz}
 \usepackage{color}
\usetikzlibrary{trees}

\newcommand{\bd}[2] {
 #1/ #2
}
\newcommand{\clos}[2] {
\langle #2; #1 \rangle
}
\newcommand*{\transname}[1]{\textsc{#1}}

\newcommand*{\transrule}[3]{
\infer[\transname{[#1]}]{#2}{#3}
}
\newcommand{\aeq}[4] {
\clos{#1}{#2} \approx \clos{#3}{#4}
}
\newcommand{\pr}[2] {
 (#1\, #2)
}
\newcommand{\app}[2] {
#1\, #2
}
\newcommand{\sframe}[5] {
#1; #2 \vdash #3 \Rightarrow #4; #5
}
\newcommand{\transition}[4] {
  $#1 \vdash #2 \stackrel{\alpha}{=} #3 \Rightarrow #4$
}
\newcommand{\swap}[2] {
  #1\leftrightarrow#2
}
\newcommand{\alphaeq}[2] {
  #1\stackrel{\alpha}{=}#2
}
\newcommand{\eqha}[2] {
  #1\stackrel{?}{=}#2
}

\title{Efficiency of a good but not linear nominal unification algorithm}
\author{Weixi Ma, Jeremy G. Siek, David Thrane Christiansen,\\ Daniel P. Friedman}

 \setbeamertemplate{footline}{%
        \includegraphics[align=c, height=0.8cm]{IU_Logo.jpg}%
        \hfill%
        \usebeamercolor[fg]{page number in head/foot}%
        \usebeamerfont{page number in head/foot}%
        \insertframenumber\,/\,\inserttotalframenumber\kern1em%
 }

\begin{document}

% \addtobeamertemplate{frametitle}{}{%
% \begin{textblock*}{100mm}(.95\textwidth,-0.5cm)
% \includegraphics[height=1cm,width=1cm]{IU_Logo.jpg}
% \end{textblock*}}


\begin{frame}
  \maketitle
\end{frame}

\begin{frame}{Nominal Unification~\citep{urban_nominal_2004}}
  
  {\centering
    $\lambda\,a.\lambda\,b.\app{X}{Y} \stackrel{?}{=} \lambda\,a.\lambda\,a.\app{a}{Z}$
    \par}
  \vspace{1cm}
  {\centering
    Unifier:\\
    $\{\bd{X}{b},\,\bd{Y}{(\swap{a}{b})\cdot Z}\}$ and $\{b\,\#\,Z\}$
    \par}
\end{frame}

\begin{frame}{Nominal Unification~\citep{urban_nominal_2004}}

  {\centering
    \begin{itemize}
    \item The fastest existing nominal unification algorithms are $O(n^2)$.
    \item The quadratic comes from swappings \\
      e.g., $\eqha{X}{(\swap{a_0}{a_0'},\,...,\swap{a_n}{a_n'})a}$.
    \item Lists cannot be replaced by structures with better lookup efficiency,
      because of $\pi\cdot(\pi'\cdot X) \equiv (\pi@\pi')\cdot X$.
    \end{itemize}
   \par}
\end{frame}

\begin{frame}{To lookup or to append?}

  {\centering
    \begin{tabular}{|c|c|c|c|}
      \hline
      & LinkedList & Hashtable & ?\\
      \hline
      lookup & $O(n)$ & $O(1)$ & $O(1)$\\
      \hline
      append & $O(1)$ & $O(n)$ & $O(1)$\\
      \hline
    \end{tabular}
  \par}
\end{frame}

\begin{frame}{It's straightforward to check $\alpha$-equivalence}{using de Bruijn indices~\citep{de_bruijn_lambda_1972}}
 
 {\centering
   $\alphaeq{\lambda\,a.\lambda\,b.a}{\lambda\,c.\lambda\,d.c}$
 \par}
\vspace{1cm}
  {\centering
  In de Bruijn indices:\par}
 {\centering
   $\alphaeq{\lambda\,\lambda\,1}{\lambda\,\lambda\,1}$
 \par}
\end{frame}

\begin{frame}{But in unification it's different}{name shadowing is lost}
 
 {\centering
   $\alphaeq{\lambda\,a.\lambda\,a.X}{\lambda\,c.\lambda\,d.c}$
 \par}
\vspace{1cm}
  {\centering
  In de Bruijn indices:\par}
 {\centering
   $\alphaeq{\lambda\,\lambda\,X}{\lambda\,\lambda\,1}$
 \par}
\end{frame}

\begin{frame}{De Bruijn indices should coexist with names}

  {\centering
    A \emph{scope} is a list of names.
  \par}
  {\centering
    A \emph{static closure} consists of {\color{blue}a term} and its enclosing scope. \\
  \par}
  \vspace{1cm}
  
  {\centering term: $\lambda\,a.\lambda\,b.{\color{blue}a}$
  \par}

{\centering
 closure: $\clos{{\color{blue}a}}{(b\,a)}$
\par}
\end{frame}

\begin{frame}{To decide $\alpha$-equivalence, we need Free and Bound}

  {\centering
  \begin{tabular}{l}
\transrule{Free}
{\Phi \vdash \texttt{Fr}\,\,a}
{a \notin \Phi} \\ \\

    \infer[\transname{[Bound]}]{\Phi \vdash \texttt{Bd}\,\,a\,\,i}{\begin{array}{@{}c@{}c@{}} (\texttt{name$\rightarrow$index}\, \Phi\,a) = i \\(\texttt{index$\rightarrow$name}\, \Phi\,i) = a
 \end{array}}

 \end{tabular}
 \par}
\end{frame}

\begin{frame}{The core of name management}

  {\centering
  \begin{tabular}{l}
    \infer[\transname{[Same-Free]}]{\aeq{a_1}{\Phi_1}{a_2}{\Phi_2}}{\begin{array}{@{}c@{}c@{}c@{}} a_1 = a_2\\\Phi_1 \vdash \texttt{Fr}\,\, a_1\\\Phi_2 \vdash \texttt{Fr}\,\, a_2
 \end{array}} \\ \\

    \infer[\transname{[Same-Bound]}]{\aeq{a_1}{\Phi_1}{a_2}{\Phi_2}}{\begin{array}{@{}c@{}c@{}c@{}} i_1 = i_2\\\Phi_1 \vdash \texttt{Bd}\,\, a_1\,\, i_1\\\Phi_2 \vdash \texttt{Bd}\,\, a_2\,\, i_2 
 \end{array}}
 \end{tabular}
 \par}
\end{frame}

\begin{frame}{Unifying name-closures}

  {\centering
    $\eqha{\lambda\,a.\lambda\,b.a}{\lambda\,c.\lambda\,d.c}$

    \vspace{1cm}
    \transition
    {\sigma}
    {\clos{a}{(b\,a)}}
    {\clos{c}{(d\,c)}}
    {\sigma}

    \vspace{1cm}
    By \transname{Same-Bound}
    \par}
\end{frame}

\begin{frame}{Unifying name-closure versus var-closure}

  {\centering
    $\eqha{\lambda\,a.\lambda\,b.X}{\lambda\,c.\lambda\,d.e}$

    \vspace{1cm}
    \transition
    {\sigma}
    {\clos{X}{(b\,a)}}
    {\clos{e}{(d\,c)}}
    {\{\bd{X}{e}\}\cup\sigma}

    \vspace{1cm}
    By \transname{Same-Free}
    \par}
\end{frame}

\begin{frame}{Unifying name-closure versus var-closure}{another example}

  {\centering
    $\eqha{\lambda\,a.\lambda\,a.X}{\lambda\,c.\lambda\,d.c}$

    \vspace{1cm}
    \transition
    {\sigma}
    {\clos{X}{(a\,a)}}
    {\clos{c}{(d\,c)}}
    \bot

    \vspace{1cm}
    No rules apply
    \par}
\end{frame}

\begin{frame}{Unifying var-closures}

 {\centering
     $\eqha{\lambda\,a.\lambda\,b.X}{\lambda\,c.\lambda\,d.Y}$

    \vspace{1cm}
   \transition
    {\sigma}
    {\clos{X}{(b\,a)}}
    {\clos{Y}{(d\,c)}}
    {?}
    \par}
   
  \vspace{1cm}
  {\centering
   $\sigma$ is not enough.
 \par}
\end{frame}

\begin{frame}{Unifying var-closures}

 {\centering
    $\eqha{\lambda\,a.\lambda\,b.X}{\lambda\,c.\lambda\,d.Y}$

    \vspace{1cm}
   \transition
    {\delta;\,\sigma}
    {\clos{X}{(b\,a)}}
    {\clos{Y}{(d\,c)}}
    {\delta';\,\sigma}

    \vspace{1cm}
    $\delta' = \{\clos{X}{(b\,a)} \stackrel{\alpha}{=} \clos{Y}{(d\,c)}\}\cup\delta$
    \par}
\end{frame}

\begin{frame}{To sum up:}
  
  \begin{itemize}
  \item De Bruijn indices coexist with names in closures.
    
  \item $\sigma$ is a substitution.
    
  \item $\delta$ contains unsolved pairs of var-closures.
    
  \item We have solved the simple case where unification variables have only been instantiated with names.

  \item Next, let's look at $\eqha{X}{\lambda\,a.Y}$ and $\eqha{X}{\app{Y}{Z}}$.
  \end{itemize}
\end{frame}

\begin{frame}{To generalize the algorithm:}

  {\centering
    Need: a shared shape for two arbitrary terms
  \par}
  
  \vspace{1cm}
  {\centering
    $\lambda\,a.X \stackrel{?}{=} \lambda\,b.\lambda\,c.b$
  \par}
\begin{figure}[H]
  \centering
  \begin{minipage}{0.45\textwidth}
    \centering
\begin{tikzpicture}
  \node{$\lambda$}
child {node {$a$}}
child {node {$X$}
  child {node {} edge from parent[draw=none]}};
  \end{tikzpicture}
  \end{minipage}\hfill
    \begin{minipage}{0.45\textwidth}
    \centering
    \begin{tikzpicture}
\node{$\lambda$}
child {node {$b$}}
child {node {$\lambda$}
  child {node {$c$}}
  child {node {b}
  }
};
    \end{tikzpicture}
    \end{minipage}
  \end{figure}
\end{frame}

\begin{frame}{To generalize the algorithm:}

  {\centering
    Need: a shared shape for two arbitrary terms
  \par}
  \vspace{1cm}
  {\centering
    $\lambda\,a.X \stackrel{?}{=} \lambda\,b.\lambda\,c.b$
  \par}
\begin{figure}[H]
  \centering
  \begin{minipage}{0.45\textwidth}
    \centering
\begin{tikzpicture}
  \node{$\lambda$}
child {node {$a$}}
child {node {$\lambda$}
  child {node {$c'$}}
  child {node {$Y$}
  }
};
  \end{tikzpicture}
  \end{minipage}\hfill
    \begin{minipage}{0.45\textwidth}
    \centering
    \begin{tikzpicture}
\node{$\lambda$}
child {node {$b$}}
child {node {$\lambda$}
  child {node {$c$}}
  child {node {$b$}
  }
};
    \end{tikzpicture}
    \end{minipage}
  \end{figure}
  
  {\centering
  Also, $\sigma = \{\bd{X}{\lambda\,c'.Y}\}\cup\sigma$
  \par}
\end{frame}

\begin{frame}{Replacing lists with hashtables}

  {\centering
    $\lambda\,a.\lambda\,b.\lambda\,b.\,\lambda\,c ...$
  \par}
  \begin{figure}[H]
  \centering
  \begin{minipage}{0.45\textwidth}
    \centering
    indices with list: \\
    \begin{tabular}{l}
      $(c\,b\,b\,a)$ \\
      \\ \\ \\
      \end{tabular}
  \end{minipage}\hfill
    \begin{minipage}{0.45\textwidth}
    \centering
    levels with two hashtables:
    \begin{tabular}{l l l}
      $a \mapsto 0$ & & $0 \mapsto a$ \\
      $b \mapsto 1$ & & $1 \mapsto b$ \\
      $b \mapsto 2$ & & $2 \mapsto b$ \\
      $c \mapsto 3$ & & $3 \mapsto c$ \\
    \end{tabular}
    \end{minipage}
  \end{figure}

\end{frame}

\begin{frame}{Where $\Phi$ changes}
\begin{tabular}{l}
  \transrule{Abs}{\sframe{\delta_0}{\sigma_0}{\alphaeq{\clos{\lambda\,a_1.t_1}{\Phi_1}}{\clos{\lambda\,a_2.t_2}{\Phi_2}}}{\delta_1}{\sigma_1}}
{%
  \sframe{\delta_0}{\sigma_0}{\alphaeq{\clos{t_1}{a_1,\Phi_1}}{\clos{t_2}{a_2,\Phi_2}}}{\delta_1}{\sigma_1}
} \\ \\ 

\transrule{App}{\sframe{\delta_0}{\sigma_0}{\alphaeq{\clos{\app{l_1}{r_1}}{{\color{blue}\Phi_1}}}{\clos{\app{l_2}{r_2}}{{\color{blue}\Phi_2}}}}{\delta_1}{\sigma_1}}
{%
\sframe{\delta_0}{\sigma_0}{\alphaeq{\clos{l_1}{{\color{blue}\Phi_1}}}{\clos{l_2}{{\color{blue}\Phi_2}}}}{\delta_0'}{\sigma_0'} \\\sframe{\delta_0'}{\sigma_0'}{\alphaeq{\clos{r_1}{{\color{blue}\Phi_1}}}{\clos{r_2}{{\color{blue}\Phi_2}}}}{\delta_1}{\sigma_1}
} \\ \\

\end{tabular}
\end{frame}

\begin{frame}{Lookup only}

  {\centering
    \begin{tabular}{|c|c|c|c|}
      \hline
      & LinkedList & Ordinary Hashtable & Persistent Hashtable \\
      \hline
      lookup & $O(n)$ & $O(1)$ & $O(log(k))$\\
      \hline
      append & $O(1)$ & $O(n)$ & whatever\\
      \hline
    \end{tabular}
  \par}
\end{frame}

\begin{frame}{Conclusion}
  
  {\centering
    \begin{itemize}
  \item Replacing swappings by scopes
  \item Replacing the need of composing by the need of sharing
  \item A $O(n \times G(n) \times log(k))$ nominal unification algorithm
    \end{itemize}
    \par}
\end{frame}
 
\begin{frame}{Primary references}
\bibliographystyle{plainnat}
\bibliography{main}
\end{frame}

\begin{frame}
  \begin{center}
    \Huge Thank you!
   \end{center}
\end{frame}
  
\end{document}
%%% Local Variables:
%%% mode: latex
%%% TeX-master: t
%%% End:
