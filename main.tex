\documentclass[a4paper,UKenglish]{lipics-v2016}

\usepackage{mathrsfs}
\usepackage{datetime}
\usepackage{nameref}
\usepackage{bm}
\usepackage{proof}
\usepackage[section]{placeins}
\usepackage{float}
\usepackage{hyperref}
\usepackage{wrapfig}
\usepackage{varwidth}

\hypersetup{
    colorlinks=true,
    linkcolor=blue,
    filecolor=magenta,      
    urlcolor=cyan,
}

\newcommand{\clos}[2] {
  \langle #1; #2 \rangle
}

\newcommand{\app}[2] {
  (#1\, #2)
}

\newcommand{\machineframe}[4] {
  #1 \lbrack #2 \rbrack_#3 \Rightarrow #4
}

\newcommand{\aframe}[2] {
  \lbrack #1, #2 \rbrack_\alpha
}

% \newcommand{\nuframe}[3] {
%   #1 \lbrack #2 \rbrack_\nu\Rightarrow #3
% }

\newcommand{\nuframe}[3] {
  #1 \vdash #2 \Rightarrow_\nu #3
}

% \newcommand{\dframe}[5] {
%   (#1, #2) \lbrack #3 \rbrack_\delta\Rightarrow (#4, #5)
% }

\newcommand{\dframe}[5] {
  (#1, #2) \vdash #3 \Rightarrow_\delta (#4, #5)
}

\newcommand{\pframe}[5] {
  (#1, #2) \vdash #3 \Rightarrow_\textrm{pull} (#4, #5)
}

% \newcommand{\pframe}[5] {
%   (#1, #2) \lbrack #3 \rbrack_\textrm{pull}\Rightarrow (#4, #5)
% }

\newcommand{\rframe}[7] {
  (#1, #2, #3) \vdash #4 \Rightarrow_\rho (#5, #6, #7)
}

\newcommand{\sframe}[7] {
  (#1, #2, #3) \vdash #4 \Rightarrow_\textrm{step} (#5, #6, #7)
}

\newcommand{\pr}[2] {
 (#1\, #2)
}

\newcommand{\bd}[2] {
 #1/ #2
}

\newcommand{\aeq}[4] {
  \clos{#1}{#2} \approx \clos{#3}{#4}
}

\newcommand*{\transname}[1]{\textsc{#1}}


\bibliographystyle{plainurl}

\title{Efficiency of a good but not linear nominal unification algorithm}

\keywords{$\alpha$-conversion; Binding operations; Efficiency; Unification}

\author[1]{Weixi Ma}
\author[2]{Jeremy G. Siek}
\author[3]{David Thrane Christiansen}
\author[4]{Daniel P. Friedman}
\affil[1]{Indiana University,
  \texttt{mvc@iu.edu}}
\affil[2]{Indiana University,
  \texttt{jsiek@indiana.edu}}
\affil[3]{Galois, Inc.,
  \texttt{dtc@galois.com}}
\affil[4]{Indiana University,
  \texttt{dfried@indiana.edu}}


\Copyright{Weixi Ma, Jeremy Siek, David Thrane Christiansen and Daniel P. Friedman}

\begin{document}

\maketitle

\begin{abstract}
  We present a nominal unification algorithm
  that runs in $O(n \times log(n) \times G(n))$ time,
  where $G$ is the functional inverse of Ackermann's function.
  Nominal unification generates a set of variable assignments,
  if there exists one,
  that makes terms involving binding operations $\alpha$-equivalent.
  We preserve names while using special representations of de Bruijn numbers.
  Operations on name handling, i.e.,
  deciding the $\alpha$-equivalence of two names and
  inferring a name that $\alpha$-equals to a given one,
  are in logarithmic time.
  To reduce an arbitrary unification problem to such name handlings,
  we preprocess the unification terms
  with the idea of Martelli-Montanari.
\end{abstract}

\section{Introduction and background}
The rules that identify terms, such as $\alpha$, $\beta$, and $\eta$
in the $\lambda$-calculus~\cite{church_calculi_1941},
are critical to building programming languages and formal systems.
As users of logic programming languages and theorem provers,
we desire such rules to be out-of-the-box in the tool-kit.
Two theories have aimed to provide this convenience:
Miller's higher-order pattern unification~\cite{miller_logic_1989} and
nominal unification~\cite{urban_nominal_2004} introduced by Urban, Pitts, and Gabbay.
Higher-order pattern unification, which handles a fragment of $\beta\eta$-rules,
is the foundation of Isabelle~\cite{paulson_natural_1986}, $\lambda$Prolog~\cite{nadathur_overview_1988}, and Twelf~\cite{pfenning_system_1999-1}.
Nominal unification, which focuses on the $\alpha$-rule,
has inspired extensions of logic programming languages, like $\alpha$Prolog~\cite{cheney_prolog:_2004} and $\alpha$Kanren~\cite{byrd_alphakanren:_2007},
as well as theorem provers, like nominal Isabelle~\cite{urban_nominal_2005} and $\alpha$LeanTAP~\cite{near_leantap:_2008}.
Although these two theories have been shown to be equally powerful~\cite{cheney_relating_2005, levy_nominal_2012},
implementing higher-order pattern unification is more complicated
because it has to deal with application and capture-avoiding substitution.
On the other hand, implementation of nominal unification,
which essentially unifies first-order terms,
is more straightforward and easier to formalize.
Beyond unification, techniques from the nominal approach,
such as swappings and freshness environments,
have impacted the areas as diverse as
rewriting~\cite{fernandez_nominal_2004, fernandez_nominal_2005, fernandez_nominal_2007, aoto_nominal_2016},
equational theories~\cite{ayala-rincon_nominal_2016},
and reasoning about bindings in abstract syntax~\cite{pitts_metalanguage_2000, gabbay_new_2002}.

When efficiency is the concern, however, nominal unification is not so practical as higher-order pattern unification.
Qian~\cite{qian_unification_1996} has shown a proof that higher-order pattern unification is decidable in linear time.
On the other hand, it has been an open problem
whether there exists a nominal unification algorithm
that can do better than $O(n^2)$.
Levy and Villaret~\cite{levy_nominal_2012} show a translation from nominal unification to higher-order pattern unification
in quadratic time while preserving the mgu.
Meanwhile, algorithmic advances by Paterson-Wegman~\cite{paterson_linear_1978} and Martelli-Montanari~\cite{martelli_efficient_1982} for unification
have inspired many improvements to the efficiency of nominal unification.
Also, the ideas like applying swappings lazily and composing swappings eagerly and sharing subterms have been explored.
Calves and Fernandez~\cite{calves_polynomial_2008, calves_complexity_2010, calves_first-order_2010}
describe quadratic
algorithms that extend Paterson-Wegman and Martelli-Montanari's
algorithms with name handling; Levy and
Villaret~\cite{levy_efficient_2010} describe a new quadratic algorithm
that reduces unification problems to a sequence of freshness and
equality constraints in linear time and then solves the constraints in quadratic time.

The inefficiency of these nominal unification algorithms
comes from the swapping actions, that is,
to decide the $\alpha$-equivalence of two names,
we need to linearly traverse a list
whose length grows with respect to the number of binders.
One might try to replace these lists with a more efficient lookup
structure, such as a hash table,
but then composing two swappings would take linear time,
and that operation is also rather frequent.
Here we present an algorithm that does not use swappings,
instead, this algorithm allows to utilize hamt~\cite{bagwell_ideal_2001},
which provides efficient lookup and uses sharing to avoid duplication.

  We organize this paper as follows.
  In section~\ref{closures}, we show a representation of
  de Bruijn numbers that is suitable for unification.
  In section~\ref{unify}, we describe the abstract machines
  for name management and unification.
  In section~\ref{efficiency}, we discuss the time complexity of this algorithm.
  The related proofs formalized in Agda,
  available at \href{https://github.com/mvcccccc/UNIF2018}{the authors' Github},
  is in progress.
  

  \section{De Bruijn numbers should coexist with names}
\label{closures}
De Bruijn numbers~\cite{de_bruijn_lambda_1972} are a technique for
representing syntax with binding structure.
A \emph{de Bruijn number} is a natural number that indicates
the distance from a name occurrence to its corresponding binder.
When all names in an expression are replaced with their corresponding de Bruijn numbers,
a direct structural equality check is sufficient to decide $\alpha$-equivalence.
A few programming languages~\cite{norell_towards_2007} use de Bruijn numbers
in their internal representations for machine manipulation during operations such as type checking.
The idea of using names for free variables and numbers for bound variables,
known as the locally nameless approach~\cite{chargueraud_locally_2012},
is employed for formalizing variable-theories~\cite{aydemir_nominal_2006, aydemir_engineering_2008}.
Also, de Bruijn numbers, combined with explicit substitution, have been introduced in
higher-order unification~\cite{dowek_higher_2000} to improve the efficiency of unification.

Despite its convenience when implementing $\alpha$-equivalence,
programs written with de Bruijn numbers are notoriously obfuscated for humans to read and understand.
What's worse, as pointed out by Berghofer and Urban~\cite{berghofer_head--head_2007},
translating pencil-and-paper style proofs to versions using de Bruijn numbers is surprisingly involved:
such translation may alter the structures of proofs.
Consequently, reproducing proofs with explicit names from de Bruijn numbers is difficult or even impossible.
Thus, for the sake of both readers and writers of proofs, it is worth providing an interface with names.

If our concern is simply deciding the $\alpha$-equivalence between expressions,
an easy way to use de Bruijn numbers while preserving names is
to traverse the expressions, annotate each name with its de Bruijn number,
then read-back the expressions without numbers.
This approach, however, does not work for unification,
because it only contains the mapping \emph{from names to numbers}.
In unification modulo $\alpha$-equivalence, one frequently needs
the mapping \emph{from numbers to names} to decide
what name to assign to a unification variable.


We propose to represent de Bruijn numbers by \emph{static closures},
hereafter referred to as \emph{closures}.
Closures preserve the mappings of both directions: names to numbers and numbers to names.


\begin{wrapfigure}{r}{0.5\textwidth}
  \begin{minipage}[b]{0.4\textwidth}
   \caption{Terms\label{table:terms}}
  \begin{tabular}{r l l l}
    $t,l,r$ & $::=$ & $a$ & atom \\
    & $|$ & $\lambda a.t$ & abstractions \\
    & $|$ & $\app{l}{r}$ & applications \\
    \end{tabular}
    \end{minipage}

\begin{minipage}[b]{0.4\textwidth}

  \caption{Free and bound\label{table:fb}}
  \begin{tabular}{l}
\infer[\transname{free}]{\Phi \vdash \texttt{Fr}\,\, a}{%
    a \notin \Phi
}\\ \\

\infer[\transname{bound}]{\Phi \vdash \texttt{Bd}\,\, a\,\, i}{%
  (\texttt{name$\rightarrow$idx}\, \Phi\, a)=i &
  (\texttt{idx$\rightarrow$name}\, \Phi\, i)=a
} \\


 \end{tabular}
\end{minipage}

\begin{minipage}[b]{0.4\textwidth}

  \caption{$\approx$-rules\label{table:same-name}}
  \begin{tabular}{l}
\infer[\transname{same-free}]{\aeq{a_1}{\Phi_1}{a_2}{\Phi_2}}{%
    \Phi_1 \vdash \texttt{Fr}\,\, a_1 &
    \Phi_2 \vdash \texttt{Fr}\,\, a_2 &
    a_1 = a_2
} \\ \\

\infer[\transname{same-bound}]{\aeq{a_1}{\Phi_1}{a_2}{\Phi_2}}{%
    \Phi_1 \vdash \texttt{Bd}\,\, a_1\,\, i_1 &
    \Phi_2 \vdash \texttt{Bd}\,\, a_2\,\, i_2 &
    i_1 = i_2
} \\


 \end{tabular}
\end{minipage}


\end{wrapfigure}

%{\renewcommand{\arraystretch}{1} 


\begin{definition}
  A closure is an ordered pair $\clos{t}{\Phi}$ of a term $t$,
  defined in figure~\ref{table:terms},
  and a scope $\Phi$,
  where the scope is an ordered list of binders that occur in the enclosing context.
  Hereafter, we refer to a name as an atom, and we refer to the atom of an abstraction as a binder.
\end{definition}

When the term of a closure is an atom, the closure itself represents a
de Bruijn number.  For example, consider the term
$\lambda\,x.\lambda\,y.x$. The de Bruijn number of the atom $x$ is $1$
and the closure-representation of this number is $\clos{x}{(y\,x)}$.
We can retrieve the number-representation by finding the
position of the first appearance of the atom in the scope. In this
case, the position of $x$ in the list $(y\,x)$ is $1$, which is its de
Bruijn number. Similarly, the de Bruijn number of $y$ is $0$.

A scope, as a list, supports three operations:
\texttt{ext-scope}, which extends the scope by adding an atom to the front of the list;
\texttt{idx$\rightarrow$atom}, which returns the atom of a given index starting from the front of the list;
and \texttt{atom$\rightarrow$idx}, which returns the location of the first appearance of a given atom counting from the front of the list.
As we are building the list in reversed order, if repeated atoms appear, the first appearance in a list shadows the others.

Now in figure~\ref{table:fb}, we can talk about free and bound variables ``constructively,''
with de Bruijn numbers serving as evidence that variables are well-scoped.
When an atom, $a$, does not appear in the scope, $\Phi$,
we say, ``$a$ is free with respect to $\Phi$,'' written as $\Phi \vdash \texttt{Fr}\, a$;
when $a$'s first appearance in $\Phi$ is the position $i$,
we say, ``$a$ is bound at $i$ with respect to $\Phi$,'' written as $\Phi \vdash \texttt{Bd}\, a\,i$.
The \transname{bound} rule has two premises to be algorithmic in both directions,
that is, given an atom we can find its index
and given an index we can find its atom, if no shadowing happens.
Figure~\ref{table:same-name} defines the rules to decide whether two atoms are $\alpha$-equivalent w.r.t theirs scopes,
written as $\aeq{a_1}{\Phi_1}{a_2}{\Phi_2}$.

\section{Unification}
\label{unify}
\begin{wrapfigure}{r}{0.5\textwidth}
  \caption{Unification terms\label{table:new-terms}}
    \begin{tabular}{r l l l}  
    $t,l,r$ & $::=$ & $a$ & atoms \\
    & $|$ & $\lambda a.t$ & abstractions \\
    & $|$ & $\app{l}{r}$ & applications \\
    & $|$ & $X$ & variables \\
    \end{tabular}
\end{wrapfigure}
In figure~\ref{table:new-terms}, we introduce unification variables, shortened as var.
First, let's consider a simplified unification problem:
a variable can only be instantiated by a name,
that is, finding the unifier of two terms
that share the same structure but differ in atoms and variables.
A unifier consists of two parts: $\sigma$ and $\delta$.
\begin{definition}
  A substitution $\sigma$ is a partial finite function from variables, $X_i$, to terms, $t_i$.
  For readability, we write $\sigma$ as a set, $\{\bd{X_1}{t_1}, ..., \bd{X_j}{t_j}\}$
  and we write $\{\bd{X}{t}\} \cup \sigma$ for extending $\sigma$ with $\bd{X}{t}$.
  For now, we assume the co-domain of a $\sigma$ only includes atoms.
\end{definition}

\begin{definition}
  We call a pair of two closures that represent that same de Bruijn number
  a \emph{closure-equation}.
  $\Delta$ is a set of closure-equations.
  We write $\Delta$ as $\{\pr{\clos{t_1}{\Phi_1}}{\clos{t_1'}{\Phi_1'}}, ..., \pr{\clos{t_i}{\Phi_1}}{\clos{t_i'}{\Phi_1'}}\}$ and
  we write $\{\pr{\clos{t}{\Phi}}{\clos{t'}{\Phi'}}\}\cup\Delta$ for extending $\Delta$ with $\pr{\clos{t}{\Phi}}{\clos{t'}{\Phi'}}$.
  We write $\delta$ as a special form of $\Delta$: for each equation in $\delta$, the term of both sides are variables.
  Given a variable $X$, we write $\delta(X)$ as retrieving the list of equations that $X$ is on one side of.
\end{definition}

\begin{wrapfigure}{r}{0.5\textwidth}
  \begin{minipage}[b]{0.4\textwidth}
  \caption{$\nu$-machine}\label{machine:nu}
  \begin{tabular}{c}
    \framebox{$\nuframe{subst}{problem_\nu^*}{subst}$} \\ \\
  \end{tabular}
  \end{minipage}

\begin{minipage}[b]{0.4\textwidth}
  \begin{tabular}{l}
    \infer[\transname{empty}]{\nuframe{\sigma_0}{\emptyset}{\sigma_0}}{%
} \\ \\

    
    \infer[\transname{name-name}]{\nuframe{\sigma_0}{(\pr{\clos{a_1}{\Phi_1}}{\clos{a_2}{\Phi_2}},\,p^*)}{\sigma_1}}{%
    \nuframe{\sigma_0}{p^*}{\sigma_1} &
    \clos{a_1}{\Phi_1} \approx \clos{a_2}{\Phi_2}                        
} \\ \\

     \infer[\transname{name-meta}]{\nuframe{\sigma_0}{(\pr{\clos{a_1}{\Phi_1}}{\clos{X_2}{\Phi_2}},\,p^*)}{\sigma_1}}{%
    \nuframe{(\{X_2/a_2\}\cup\sigma_0)}{p^*}{\sigma_1} &
    \clos{a_1}{\Phi_1} \approx \clos{a_2}{\Phi_2}                        
} \\

  \end{tabular}
  \end{minipage}
  
  \begin{minipage}[b]{\textwidth}
  \caption{$\delta$-machine and pull}\label{machine:delta}
  \begin{tabular}{c}
    \fbox{\begin{varwidth}{\textwidth}
        $\dframe{subst}{problem_\delta^*}{var^*}{subst}{problem_\delta^*}$ \\
        $\pframe{subst}{var^*}{problem_\delta^*}{subst}{var^*}$
        \end{varwidth}} \\ \\
  \end{tabular}
  \end{minipage}

\begin{minipage}[b]{0.4\textwidth}
  \begin{tabular}{l}
         \infer[\transname{empty-q}]{\dframe{\sigma_0}{\delta_0}{\emptyset}{\sigma_0}{\delta_0}}{%
} \\ \\

         \infer[\transname{empty-D}]{\dframe{\sigma_0}{\emptyset}{q}{\sigma_0}{\emptyset}}{%
} \\ \\

        \infer[\transname{step}]{\dframe{\sigma_0}{\delta_0}{(X, q)}{\sigma_1}{\delta_1}}{%
\dframe{\sigma_0'}{\delta_0 \setminus \delta_0(X)}{q'}{\sigma_1}{\delta_1} \\
                                                        \pframe{\sigma_0}{q}{\delta_0(X)}{\sigma_0'}{q'}
} \\ \\


             \infer[\transname{empty}]{\pframe{\sigma_0}{q_0}{\emptyset}{\sigma_0}{q_0}}{%
} \\ \\

                 \infer[\transname{name-name}]{\pframe{\sigma_0}{q_0}{(\pr{\clos{X_1}{\Phi_1}}{\clos{X_2}{\Phi_2}}, p^*)}{\sigma_1}{q_1}}{%
    \pframe{\sigma_0}{q_0}{p^*}{\sigma_1}{q_1} \\
    \aeq{a_1}{\Phi_1}{a_2}{\Phi_2} & \sigma_0(X_1) = a_1 & \sigma_0(X_2) = a_2 
} \\ \\

                     \infer[\transname{name-meta}]{\pframe{\sigma_0}{q_0}{(\pr{\clos{X_1}{\Phi_1}}{\clos{X_2}{\Phi_2}}, p^*)}{\sigma_1}{q_1}}{%
    \pframe{\{X_2/a_2\}\cup\sigma_0}{(X_2, q_0)}{p^*}{\sigma_1}{q_1} \\
    \aeq{a_1}{\Phi_1}{a_2}{\Phi_2} \\ \sigma_0(X_1) = a_1 \\ X_2\notin dom(\sigma_0)
} 

  \end{tabular}

  \end{minipage}
  \end{wrapfigure}

The simplified problem is about solving three kinds of closure-equations:
name-name, name-var, and var-var.
We refer to the name-name and name-var problem as problem$_\nu$
and refer to the var-var problem as problem$_\delta$.
Given a set of problem$_\nu$ and problem$_\delta$,
we first run the $\nu$-machine, defined in figure~\ref{machine:nu}, on the $\nu$-problems to generate a substitution.
Then the $\delta$-machine computes the final unifier on three inputs:
the substitution from $\nu$,
the var-var problems, $\delta$,
and a list of known variables, which is initialized by the domain of the substitution.
If no transitions apply, the machine fails and the unification problem has no unifier.


  \begin{lemma}\label{lemma:numachine}
 For all finite $p^*$ and $\delta$, the $\nu$-machine and the $\delta$-machine terminates;
    for all input, the $\nu$-machine and the $\delta$-machine succeeds with the mgu
    if and only if there exists one.
    \end{lemma}
    \begin{proof}
     By structural induction on the transitions of the machines.
    \end{proof}

    Now the question is how to generalize the previous algorithm, that is,
    given two arbitrary terms, where a variable may be instantiated by any term besides atoms,
    can we re-shape the two terms to create a proper input to $\nu$ and $\delta$?

    Finding the common structure, obviously, is merely a first-order unification problem.
    The $\rho$-machine, defined in figure~\ref{table:rmachine}, adapts the idea of Martelli-Montanari
    and reduces an arbitrary nominal unification problem to a $p^*$, a $\delta$, and a substitution.

      %\begin{wrapfigure}{l}{0.5\textwidth}
    \begin{figure}
  \caption{$\rho$-machine}\label{table:rmachine}
   \begin{minipage}[b]{\textwidth}
  \begin{tabular}{c}
    \fbox{\begin{varwidth}{\textwidth}
        $\rframe{problem_\nu^*}{problem_\delta^*}{subst}{multieqn^*}{problem_\nu^*}{problem_\delta^*}{subst}$ \\
        $\sframe{problem_\nu^*}{problem_\delta^*}{subst}{multieqn}{problem_\nu^*}{problem_\delta^*}{subst}$ \\
        \end{varwidth}} \\ \\
  \end{tabular}
  \end{minipage}

  \begin{minipage}[b]{0.4\textwidth}
  \begin{tabular}{l}
    \infer[\transname{empty}]{\rframe{p_0}{\delta_0}{\sigma_0}{\emptyset}{p_0}{\delta_0}{\sigma_0}}{%
    } \\  \\

    \infer[\transname{step}]{\rframe{p_0}{\delta_0}{\sigma_0}{(U, U^*)}{p_1}{\delta_1}{\sigma_1}}{%
    \rframe{p_0'}{\delta_0'}{\sigma_0'}{U^*}{p_1}{\delta_1}{\sigma_1} \\
    \sframe{p_0}{\delta_0}{\sigma_0}{U}{p_0'}{\delta_0'}{\sigma_0'}
    } \\ \\

        \infer[\transname{name-name}]{\sframe{p_0}{\delta_0}{\sigma_0}{\pr{\clos{a_1}{\Phi_1}}{\clos{a_2}{\Phi_2}}}{p_1}{\delta_0}{\sigma_0}}{%
    p_1 = \pr{\clos{a_1}{\Phi_1}}{\clos{a_2}{\Phi_2}} \cup p_0
    } \\ \\
        \infer[\transname{name-meta}]{\sframe{p_0}{\delta_0}{\sigma_0}{\pr{\clos{a_1}{\Phi_1}}{\clos{X_2}{\Phi_2}}}{p_1}{\delta_0}{\sigma_0}}{%
    p_1 = \pr{\clos{a_1}{\Phi_1}}{\clos{X_2}{\Phi_2}} \cup p_0
    } \\ \\
        \infer[\transname{meta-meta}]{\sframe{p_0}{\delta_0}{\sigma_0}{\pr{\clos{a_1}{\Phi_1}}{\clos{X_2}{\Phi_2}}}{p_0}{\delta_1}{\sigma_0}}{%
    \delta_1 = \pr{\clos{X_1}{\Phi_1}}{\clos{X_2}{\Phi_2}} \cup \delta_0
    } \\ \\
        \infer[\transname{app-app}]{\sframe{p_0}{\delta_0}{\sigma_0}{\pr{\clos{\app{l_1}{r_1}}{\Phi_1}}{\clos{\app{l_2}{r_2}}{\Phi_2}}}{p_1}{\delta_1}{\sigma_1}}{%
    \sframe{p_0}{\delta_0}{\sigma_0}{\pr{\clos{l_1}{\Phi_1}}{\clos{l_2}{\Phi_2}}}{p_0'}{\delta_0'}{\sigma_0'} \\
    \sframe{p_0'}{\delta_0'}{\sigma_0'}{\pr{\clos{r_1}{\Phi_1}}{\clos{r_2}{\Phi_2}}}{p_1}{\delta_1}{\sigma_1}
    } \\ \\
            \infer[\transname{abs-abs}]{\sframe{p_0}{\delta_0}{\sigma_0}{\pr{\clos{\lambda\,a_1.t_1}{\Phi_1}}{\clos{\lambda\,a_2.t_2}{\Phi_2}}}{p_1}{\delta_1}{\sigma_1}}{%
    \sframe{p_0}{\delta_0}{\sigma_0}{\pr{\clos{t_1}{\Phi_1'}}{\clos{t_2}{\Phi_2'}}}{p_1}{\delta_1}{\sigma_1}\\
    \Phi_1' = (\texttt{ext-scope}\, \Phi_1\, a_1) & \Phi_2' = (\texttt{ext-scope}\, \Phi_2\, a_2)
    } \\ \\
        \infer[\transname{meta-app}]{\sframe{p_0}{\delta_0}{\sigma_0}{\pr{\clos{X_1}{\Phi_1}}{\clos{\app{l_2}{r_2}}{\Phi_2}}}{p_1}{\delta_1}{\sigma_1}}{%
    \sframe{p_0}{\delta_0}{\{X_1/(X_l, X_r)\}\cup\sigma_0'}{\pr{\clos{X_l}{\Phi_1}}{\clos{l_2}{\Phi_2}}}{p_0'}{\delta_0'}{\sigma_0'} \\
    \sframe{p_0'}{\delta_0'}{\sigma_0'}{\pr{\clos{X_r}{\Phi_1}}{\clos{r_2}{\Phi_2}}}{p_1}{\delta_1}{\sigma_1} 
    } \\ \\
        \infer[\transname{meta-abs}]{\sframe{p_0}{\delta_0}{\sigma_0}{\pr{\clos{X_1}{\Phi_1}}{\clos{\lambda\,a_2.t_2}{\Phi_2}}}{p_1}{\delta_1}{\sigma_1}}{%
    \sframe{p_0}{\delta_0}{\bd{X_1}{\lambda\,a_1.X_t}\cup\sigma_0'}{\pr{\clos{X_t}{\Phi_1'}}{\clos{t_2}{\Phi_2'}}}{p_1}{\delta_1}{\sigma_1} \\
    \Phi_1' = (\texttt{ext-scope}\, \Phi_1\, a_1) & \Phi_2' = (\texttt{ext-scope}\, \Phi_2\, a_2) 
    } \\ \\



    % \transname{empty} & $\rframe{p_0}{\delta_0}{\sigma_0}{\emptyset}{p_0}{\delta_0}{\sigma_0}$ \\
    % \transname{step} & $\rframe{p_0}{\delta_0}{\sigma_0}{(U,\,U^*)}{p_1}{\delta_1}{\sigma_1}$ \\
    %                       & where $\rframe{p_0'}{\delta_0'}{\sigma_0'}{U*}{p_1}{\delta_1}{\sigma_1}$ \\
    %                       & \hspace{8mm} $\cframe{p_0}{\delta_0}{\sigma_0}{U}{p_0'}{\delta_0'}{\sigma_0'}$ \\
    %                   & \\
    % \transname{name-name} & $\cframe{p_0}{\delta_0}{\sigma_0}{\pr{\clos{a_1}{\Phi_1}}{\clos{a_2}{\Phi_2}}}{p_0'}{\delta_0}{\sigma_0}$ \\
   %                        & where $p_0' = {\pr{\clos{a_1}{\Phi_1}}{\clos{a_2}{\Phi_2}}} \cup p_0$  \\
   %  \transname{name-meta} & $\cframe{p_0}{\delta_0}{\sigma_0}{\pr{\clos{a_1}{\Phi_1}}{\clos{X_2}{\Phi_2}}}{p_0'}{\delta_0}{\sigma_0}$ \\
   %                        & where $p_0' = {\pr{\clos{a_1}{\Phi_1}}{\clos{X_2}{\Phi_2}}} \cup p_0$  \\
   %  \transname{meta-meta} & $\cframe{p_0}{\delta_0}{\sigma_0}{\pr{\clos{X_1}{\Phi_1}}{\clos{X_2}{\Phi_2}}}{p_0}{\delta_0'}{\sigma_0}$ \\
   %                        & where $p_0' = {\pr{\clos{X_1}{\Phi_1}}{\clos{X_2}{\Phi_2}}} \cup p_0$  \\
   %  \transname{app-app} & $\cframe{p_0}{\delta_0}{\sigma_0}{\pr{\clos{\app{l_1}{r_1}}{\Phi_1}}{\clos{\app{l_2}{r_2}}{\Phi_2}}}{p_1}{\delta_1}{\sigma_1}$ \\ 
   %  & where $\cframe{p_0}{\delta_0}{\sigma_0}{\pr{\clos{l_1}{\Phi_1}}{\clos{l_2}{\Phi_2}}}{p_0'}{\delta_0'}{\sigma_0'}$ \\
   %  & \hspace{8mm} $\cframe{p_0'}{\delta_0'}{\sigma_0'}{\pr{\clos{r_1}{\Phi_1}}{\clos{r_2}{\Phi_2}}}{p_1}{\delta_1}{\sigma_1}$ \\
   %  \transname{abs-abs} & $\cframe{p_0}{\delta_0}{\sigma_0}{\pr{\clos{\lambda\,a_1.t_1}{\Phi_1}}{\clos{\lambda\,a_2.t_2}{\Phi_2}}}{p_1}{\delta_1}{\sigma_1}$ \\ 
   %  & where $\cframe{p_0}{\delta_0}{\sigma_0}{\pr{\clos{t_1}{\Phi_1'}}{\clos{t_2}{\Phi_2'}}}{p_1}{\delta_1}{\sigma_1}$ \\
   %  & \hspace{8mm} $\Phi_1' = (\texttt{ext-scope}\, \Phi_1\, a_1)$ \\
   %  & \hspace{8mm} $\Phi_2' = (\texttt{ext-scope}\, \Phi_2\, a_2)$ \\
   %  \transname{meta-app} & $\cframe{p_0}{\delta_0}{\sigma_0}{\pr{\clos{X_1}{\Phi_1}}{\clos{\app{l_2}{r_2}}{\Phi_2}}}{p_1}{\delta_1}{\sigma_1}$ \\ 
   %  & where $\cframe{p_0}{\delta_0}{\sigma_0'}{\pr{\clos{X_l}{\Phi_1}}{\clos{l_2}{\Phi_2}}}{p_0'}{\delta_0'}{\sigma_0''}$ \\
   %  & \hspace{8mm} $\cframe{p_0'}{\delta_0'}{\sigma_0''}{\pr{\clos{X_r}{\Phi_1}}{\clos{r_2}{\Phi_2}}}{p_1}{\delta_1}{\sigma_1}$ \\
   %  & \hspace{8mm} $\sigma_0' = \{\bd{X_1}{\app{X_l}{X_r}}\} \cup \sigma_0$ \\
   % \transname{meta-abs} & $\cframe{p_0}{\delta_0}{\sigma_0}{\pr{\clos{X_1}{\Phi_1}}{\clos{\lambda\,a_2.t_2}{\Phi_2}}}{p_1}{\delta_1}{\sigma_1}$ \\ 
   %  & where $\cframe{p_0}{\delta_0}{\sigma_0'}{\pr{\clos{X_t}{\Phi_1'}}{\clos{t_2}{\Phi_2'}}}{p_1}{\delta_1}{\sigma_1}$ \\
   %  & \hspace{8mm} $\Phi_1' = (\texttt{ext-scope}\, \Phi_1\, a_1)$ \\
   %  & \hspace{8mm} $\Phi_2' = (\texttt{ext-scope}\, \Phi_2\, a_2)$ \\
   %  & \hspace{8mm} $\sigma_0' = \{\bd{X_1}{\lambda\,a_1.X_t}\} \cup \sigma_0$ \\

  \end{tabular}
  \end{minipage}
\end{figure}

\section{A note on time complexity}
    \label{efficiency}
    In the previous sections, we represent scopes by lists for simplicity,
    but lists are inefficient for variable lookup.
    To have better time complexity,
    we represent a scope with a counter and two immutable hashtables, also known as hamt~\cite{bagwell_ideal_2001}.
    One hashtable maps from names to numbers,
    the other maps from numbers to names,
    and the counter is used to track the de Bruijn number.
    When we extend a scope with a name,
    we extend the two hashtables with the corresponding maps and add one to the counter.
    An immutable hashtable, in practice, has constant time for update and lookup,
    thought the worst case scenario could be $O(log(n))$.
    Thus, \texttt{ext-scope}, \texttt{idx$\rightarrow$atom}, and \texttt{atom$\rightarrow$idx}
    are all logarithmic time.
    In addition, using immutable structures avoids copying the entire data-structure
    when branching, in particular, during the \transname{app-app} rule of the $\rho$-machine.

    We implement $\delta$ with a hashtable that maps from a variable
    to the list that contains its closure-equations.
    Doing so doubles the space, i.e, the equation $\aeq{X}{\Phi_1}{Y}{\Phi_2}$ exists in
    both $X$'s entry and $Y$'s entry, but improves the time efficiency.

    Given this optimization, the $\nu$-machine and the $\delta$-machine
    are both linear w.r.t the size of the problems.
    The $\rho$-machine, as shown by Martelli and Montanari, is $O(n \times G(n))$,
    when representing sets with UNION-FIND~\cite{tarjan_efficiency_1975},
    where $n$ is the number of variable occurrences in the original terms.

    \clearpage

\bibliography{main}

\end{document}
%%% Local Variables:
%%% mode: latex
%%% TeX-master: t
%%% End:
