\documentclass[a4paper,UKenglish]{lipics-v2016}

% testing

\usepackage{mathrsfs}
\usepackage{datetime}
\usepackage{nameref}
\usepackage{bm}
\usepackage{proof}
\usepackage[section]{placeins}
\usepackage{float}
\usepackage{hyperref}
\usepackage{wrapfig}
\usepackage{varwidth}
\usepackage{color}
\usepackage{natbib}

\hypersetup{
    colorlinks=true,
    linkcolor=blue,
    filecolor=magenta,      
    urlcolor=cyan,
}

\newcommand{\clos}[2] {
  \langle #1; #2 \rangle
}

\newcommand{\app}[2] {
  (#1\, #2)
}

\newcommand{\machineframe}[4] {
  #1 \lbrack #2 \rbrack_#3 \Rightarrow #4
}

\newcommand{\aframe}[2] {
  \lbrack #1, #2 \rbrack_\alpha
}

% \newcommand{\nuframe}[3] {
%   #1 \lbrack #2 \rbrack_\nu\Rightarrow #3
% }

\newcommand{\nuframe}[3] {
  #1 \vdash #2 \Rightarrow_\nu #3
}

% \newcommand{\dframe}[5] {
%   (#1, #2) \lbrack #3 \rbrack_\delta\Rightarrow (#4, #5)
% }

\newcommand{\dframe}[5] {
  #1; #2 \vdash #3 \Rightarrow_\delta #4; #5
}

\newcommand{\pframe}[5] {
  #1; #2 \vdash #3 \Rightarrow_\textrm{pull} #4; #5
}

% \newcommand{\pframe}[5] {
%   (#1, #2) \lbrack #3 \rbrack_\textrm{pull}\Rightarrow (#4, #5)
% }

\newcommand{\rframe}[7] {
  (#1, #2, #3) \vdash #4 \Rightarrow_\rho (#5, #6, #7)
}

\newcommand{\sframe}[7] {
  (#1, #2, #3) \vdash #4 \Rightarrow_\textrm{step} (#5, #6, #7)
}

\newcommand{\pr}[2] {
 (#1\, #2)
}

\newcommand{\eq}[2] {
 #1 = #2
}

\newcommand{\bd}[2] {
 #1/ #2
}

\newcommand{\aeq}[4] {
  \clos{#1}{#2} \approx \clos{#3}{#4}
}

\newcommand*{\transname}[1]{\textsc{#1}}


%\bibliographystyle{plainurl}
\bibliographystyle{plainnat}

\title{Efficiency of a good but not linear nominal unification algorithm}

\keywords{$\alpha$-conversion; Binding operations; Efficiency; Unification}

\author[1]{Weixi Ma}
\author[2]{Jeremy G. Siek}
\author[3]{David Thrane Christiansen}
\author[4]{Daniel P. Friedman}
\affil[1]{Indiana University,
  \texttt{mvc@iu.edu}}
\affil[2]{Indiana University,
  \texttt{jsiek@indiana.edu}}
\affil[3]{Galois, Inc.,
  \texttt{dtc@galois.com}}
\affil[4]{Indiana University,
  \texttt{dfried@indiana.edu}}


\Copyright{Weixi Ma, Jeremy Siek, David Thrane Christiansen and Daniel P. Friedman}

\begin{document}

\maketitle

\begin{abstract}
  We present a nominal unification algorithm that runs in $O(n \times
  log(n) \times G(n))$ time, where $G$ is the functional inverse of
  Ackermann's function.  Nominal unification generates a set of
  variable assignments, if there exists one, that makes terms
  involving binding operations $\alpha$-equivalent. We preserve names
  while using special representations of de Bruijn numbers
  to enable efficient name management.
  % {\color{red}[The following sentence is not clear enough. --Jeremy]}
  We use Martelli-Montanari style multi-equation reduction
  to generate the name management problems from arbitrary unification terms.
  % {\color{red} The following phrase needs to be reworded.  ``the''
  %   indicates exactly one. But they had many ideas.  Which one are you
  %   referring to? Be specific.}  with the idea of Martelli-Montanari.
\end{abstract}

\section{Introduction and background}

The rules that identify terms, such as $\alpha$, $\beta$, and $\eta$
in the $\lambda$-calculus~\citep{church_calculi_1941}, are critical to
building programming languages and formal systems.  As users of logic
programming languages and theorem provers, we desire such rules to be
out-of-the-box in the tool-kit.  Two theories have aimed to provide
this convenience: higher-order pattern unification of
\citet{miller_logic_1989} and nominal unification of
\citet{urban_nominal_2004}.  Higher-order pattern unification, which
handles a fragment of $\beta\eta$-rules, is the foundation of
Isabelle~\citep{paulson_natural_1986},
$\lambda$Prolog~\citep{nadathur_overview_1988}, and
Twelf~\citep{pfenning_system_1999-1}.  Nominal unification, which
focuses on the $\alpha$-rule, has inspired extensions of logic
programming languages, like $\alpha$Prolog~\citep{cheney_prolog:_2004}
and $\alpha$Kanren~\citep{byrd_alphakanren:_2007}, as well as theorem
provers, like nominal Isabelle~\citep{urban_nominal_2005} and
$\alpha$LeanTAP~\citep{near_leantap:_2008}.  Although these two
theories can be reduced to eachother~\citep{cheney_relating_2005,
  levy_nominal_2012}, implementing higher-order pattern unification is
more complicated because it has to deal with application and
capture-avoiding substitution.  On the other hand, implementation of
nominal unification, which essentially unifies first-order terms, is
more straightforward and easier to formalize.  Beyond unification,
techniques from the nominal approach, such as swapping and freshness
environments, have impacted the areas as diverse as
rewriting~\citep{fernandez_nominal_2004, fernandez_nominal_2005,
  fernandez_nominal_2007, aoto_nominal_2016}, equational
theories~\citep{ayala-rincon_nominal_2016}, and reasoning about
bindings in abstract syntax~\citep{pitts_varlanguage_2000,
  gabbay_new_2002}.

Concerning time complexity, \citet{qian_unification_1996} prove that
higher-order pattern unification is decidable in linear time.  On the
other hand, it has been an open problem whether there exists a nominal
unification algorithm that can do better than
$O(n^2)$. \citet{levy_nominal_2012} give a quadratic time reduction
from nominal unification to higher-order pattern unification.
Meanwhile, algorithmic advances by \citet{paterson_linear_1978} and
\citet{martelli_efficient_1982} for unification have inspired many
improvements to the efficiency of nominal unification.  Also, ideas
like applying swappings lazily and composing swappings eagerly and
sharing subterms have been explored. \citet{calves_complexity_2010}
describes quadratic algorithms that extend Paterson-Wegman and
Martelli-Montanari's algorithms with name handling;
\citet{levy_efficient_2010} describe a quadratic algorithm that
reduces unification problems to a sequence of freshness and equality
constraints and then solves the constraints.

The inefficiency of these nominal unification algorithms comes from
the swapping actions, that is, to decide the $\alpha$-equivalence of
two names, we need to linearly traverse a list whose length grows with
respect to the number of binders.  One might try to replace these
lists with a more efficient lookup structure, such as a hash table,
but then composing two swappings would take linear time, and that
operation is also rather frequent.  Here we present an algorithm that
does not use swappings but instead represents names with De Bruijn
numbers which enables the use of persistent hash tables, in
particular, a Hash Array Mapped Trie
(HAMT)~\citep{bagwell_ideal_2001}. HAMTs provide efficient lookup and
they use sharing to avoid the linear-time costs that would normally be
associated with duplicating a hash table.

We organize this paper as follows.  In section~\ref{closures}, we show
an alternative representation of de Bruijn numbers that is suitable
for unification.  In section~\ref{unify}, we describe the abstract
machines for name management and unification.  In
section~\ref{efficiency}, we discuss the time complexity of this
algorithm. The proofs of our claims are in progress and are available
at \href{https://github.com/mvcccccc/UNIF2018}{the authors' Github},
formalized in Agda.
  

\section{De Bruijn numbers should coexist with names}
\label{closures}

De Bruijn numbers are a technique for representing syntax with binding
structure~\citep{de_bruijn_lambda_1972}.  A \emph{de Bruijn number} is
a natural number that indicates the distance from a name occurrence to
its corresponding binder.  When all names in an expression are
replaced with their corresponding de Bruijn numbers, a direct
structural equality check is sufficient to decide
$\alpha$-equivalence.  A few programming
languages~\citep{norell_towards_2007} use de Bruijn numbers in their
internal representations for machine manipulation during operations
such as type checking.  The idea of using names for free variables and
numbers for bound variables, known as the locally nameless
approach~\citep{chargueraud_locally_2012}, is employed for formalizing
variable-theories~\citep{aydemir_nominal_2006,
  aydemir_engineering_2008}.  Also, de Bruijn numbers, combined with
explicit substitution, have been introduced in higher-order
unification~\citep{dowek_higher_2000} to improve the efficiency of
unification.

Despite its convenience when implementing $\alpha$-equivalence,
programs written with de Bruijn numbers are notoriously obfuscated for
humans to read and understand.  What's worse, as pointed out by
~\citet{berghofer_head--head_2007}, translating
pencil-and-paper style proofs to versions using de Bruijn numbers is
surprisingly involved: such translation may alter the structures of
proofs.  Consequently, reproducing proofs with explicit names from de
Bruijn numbers is difficult or even impossible.  Thus, for the sake of
both readers and writers of proofs, it is worth providing an interface
with names.

If our concern is simply deciding the $\alpha$-equivalence between
expressions, an easy way to use de Bruijn numbers while preserving
names is to traverse the expressions, annotate each name with its de
Bruijn number, then read-back the expressions without numbers.  This
approach, however, does not work for unification, because it only
contains the mapping \emph{from names to numbers}.  In unification
modulo $\alpha$-equivalence, one frequently needs the mapping
\emph{from numbers to names} to decide what name to assign to a
unification variable.

We represent de Bruijn numbers by \emph{static closures},
hereafter referred to as \emph{closures}.  Closures preserve the
mappings in both directions: names to numbers and numbers to names.


\begin{wrapfigure}{r}{0.5\textwidth}
  \begin{minipage}[b]{0.4\textwidth}
   \caption{Terms\label{table:terms}}
  \begin{tabular}{r l l l}
    $t,l,r$ & $::=$ & $a$ & atom \\
    & $|$ & $\lambda a.t$ & abstractions \\
    & $|$ & $\app{l}{r}$ & applications \\
    \end{tabular}
    \end{minipage}

\begin{minipage}[b]{0.4\textwidth}

  \caption{Free and bound\label{table:fb}}
  \begin{tabular}{l}
\infer[\transname{free}]{\Phi \vdash \texttt{Fr}\,\, a}{%
    a \notin \Phi
}\\ \\

\infer[\transname{bound}]{\Phi \vdash \texttt{Bd}\,\, a\,\, i}{%
  (\texttt{atom$\rightarrow$idx}\, \Phi\, a)=i &
  (\texttt{idx$\rightarrow$atom}\, \Phi\, i)=a
} \\


 \end{tabular}
\end{minipage}

\begin{minipage}[b]{0.4\textwidth}

  \caption{$\approx$-rules\label{table:same-atom}}
  \begin{tabular}{l}
\infer[\transname{same-free}]{\aeq{a_1}{\Phi_1}{a_2}{\Phi_2}}{%
    \Phi_1 \vdash \texttt{Fr}\,\, a_1 &
    \Phi_2 \vdash \texttt{Fr}\,\, a_2 &
    a_1 = a_2
} \\ \\

\infer[\transname{same-bound}]{\aeq{a_1}{\Phi_1}{a_2}{\Phi_2}}{%
    \Phi_1 \vdash \texttt{Bd}\,\, a_1\,\, i_1 &
    \Phi_2 \vdash \texttt{Bd}\,\, a_2\,\, i_2 &
    i_1 = i_2
} \\


 \end{tabular}
\end{minipage}


\end{wrapfigure}

%{\renewcommand{\arraystretch}{1} 


\begin{definition}
  A \emph{closure} is an ordered pair $\clos{t}{\Phi}$ of a term $t$,
  defined in figure~\ref{table:terms}, and a scope $\Phi$, where the
  scope is an ordered list of names for the binders in the enclosing
  context.  Hereafter, we refer to a name as an \emph{atom}.
\end{definition}

When the term of a closure is an atom, the closure itself represents a
de Bruijn number.  For example, consider the term
$\lambda\,x.\lambda\,y.x$. The de Bruijn number of the atom $x$ is $1$
and the closure-representation of this number is $\clos{x}{(y\,x)}$.
We can retrieve the number-representation by finding the position of
the first appearance of the atom in the scope. In this case, the
position of $x$ in the list $(y\,x)$ is $1$, which is its de Bruijn
number. Similarly, the de Bruijn number of $y$ is $0$.

A scope, as a list, supports three operations: \texttt{ext-scope},
which extends the scope by adding an atom to the front of the list;
\texttt{idx$\rightarrow$atom}, which returns the atom of a given index
starting from the front of the list; and
\texttt{atom$\rightarrow$idx}, which returns the location of the first
appearance of a given atom counting from the front of the list.  As we
are building the list in reversed order, if repeated atoms appear, the
first appearance in a list shadows the others.

Now in figure~\ref{table:fb}, we can talk about free and bound
variables ``constructively,'' with de Bruijn numbers serving as
evidence that variables are well-scoped.  When an atom, $a$, does not
appear in the scope, $\Phi$, we say, ``$a$ is free with respect to
$\Phi$,'' written as $\Phi \vdash \texttt{Fr}\, a$; when $a$'s first
appearance in $\Phi$ is the position $i$, we say, ``$a$ is bound at
$i$ with respect to $\Phi$,'' written as $\Phi \vdash \texttt{Bd}\,
a\,i$.  The \transname{bound} rule has two premises to be algorithmic
in both directions, that is, given an atom we can find its index and
given an index we can find its atom, if no shadowing happen.
Figure~\ref{table:same-name} defines the rules to decide whether two
atoms are $\alpha$-equivalent w.r.t theirs scopes, written as
$\aeq{a_1}{\Phi_1}{a_2}{\Phi_2}$.

% temporary
\pagebreak

\section{Unification}

\label{unify}
\begin{wrapfigure}{r}{0.5\textwidth}
  \caption{Unification terms and problems\label{table:new-terms}}
    \begin{tabular}{r l l l}  
    $X,Y$   &       &     & variables \\
      $xs,ys$ & $::=$ & $\epsilon$& list of variables \\
      & $|$&  $X,xs$ &  \\
    $t,l,r$ & $::=$ & $a$ & atoms \\
    & $|$ & $\lambda a.t$ & abstractions \\
    & $|$ & $\app{l}{r}$ & applications \\
      & $|$ & $X$ & \\
      $e_\nu$  & $::=$ & $(a,\Phi) = (a,\Phi)$ & $\nu$-equations \\
           & $|$ &  $(X,\Phi) = (a,\Phi)$ \\
      $p_\nu$  & $::=$ & $\epsilon$ & $\nu$-problems \\
           & $|$ & $e_\nu, p_\nu$ \\
      $e_\delta$  & $::=$& $(X,\Phi) = (X,\Phi)$ & $\delta$-equations \\
      $p_\delta$  & $::=$ & $\epsilon$ & $\delta$-problems \\
           & $|$ & $e_\delta, p_\delta$
    \end{tabular}
\end{wrapfigure}

In figure~\ref{table:new-terms}, we introduce unification variables,
shortened as var.  First, let's consider a simplified unification
problem: a variable can only be instantiated by an atom, that is,
finding the unifier of two terms that share the same structure but
differ in atoms and variables.  A unifier consists of two parts:
$\sigma$ and $\delta$.
\begin{definition}
  A substitution $\sigma$ is a partial finite function from variables,
  $X_i$, to terms, $t_i$.  For readability, we write $\sigma$ as a
  set, $\{\bd{X_1}{t_1}, ..., \bd{X_j}{t_j}\}$ and we write
  $\{\bd{X}{t}\} \cup \sigma$ for extending $\sigma$ with $\bd{X}{t}$.
  For the simplified problem, we restrict that $t$ can only be an atom.
\end{definition}

\begin{definition}
  A closure equation is a pair of two closures that are
  $\alpha$-equivalent.  $\Delta$ is a set of closure-equations.  We
  write $\Delta$ as $\{\pr{\clos{t_1}{\Phi_1}}{\clos{t_1'}{\Phi_1'}},
  ..., \pr{\clos{t_i}{\Phi_1}}{\clos{t_i'}{\Phi_1'}}\}$ and we write
  $\{\pr{\clos{t}{\Phi}}{\clos{t'}{\Phi'}}\}\cup\Delta$ for extending
  $\Delta$ with $\pr{\clos{t}{\Phi}}{\clos{t'}{\Phi'}}$.  We write
  $\delta$ as a special form of $\Delta$: for each equation in
  $\delta$, the term of both sides are variables.  Given a variable
  $X$, we write $\delta(X)$ as retrieving the list of
  closure-equations that $X$ is on one side of.
\end{definition}

\begin{wrapfigure}{r}{0.5\textwidth}
  \begin{minipage}[b]{0.4\textwidth}
  \caption{$\nu$-machine}\label{machine:nu}
  \begin{tabular}{c}
    \framebox{$\nuframe{\sigma}{p_\nu}{\sigma}$} \\ \\
  \end{tabular}
  \end{minipage}

\begin{minipage}[b]{0.4\textwidth}
  \begin{tabular}{l}
    \infer[\transname{Empty}]{\nuframe{\sigma_0}{\epsilon}{\sigma_0}}{%
} \\ \\

    
    \infer[\transname{Name-Name}]{\nuframe{\sigma_0}{\eq{\clos{a_1}{\Phi_1}}{\clos{a_2}{\Phi_2}},\,p}{\sigma_1}}{%
    \nuframe{\sigma_0}{p}{\sigma_1} &
    \clos{a_1}{\Phi_1} \approx \clos{a_2}{\Phi_2}                        
} \\ \\

     \infer[\transname{Name-Var}]{\nuframe{\sigma_0}{\eq{\clos{a_1}{\Phi_1}}{\clos{X_2}{\Phi_2}},\,p}{\sigma_1}}{%
    \nuframe{\{X_2/a_2\}\cup\sigma_0}{p}{\sigma_1} \\
    \clos{a_1}{\Phi_1} \approx \clos{a_2}{\Phi_2}                        
} \\

  \end{tabular}
  \end{minipage}
\end{wrapfigure}


The simplified problem is about solving three kinds of
closure-equations: atom-atom, atom-var, and var-var.
{\color{red}[ talk about the equations and problems
    defined in Figure~\ref{table:new-terms} ]}
We refer to a
atom-atom or atom-var problem as a \emph{problem$_\nu$} and refer to a
var-var problem as a \emph{problem$_\delta$}.
Given a set of
problem$_\nu$ and a set of problem$_\delta$, we first run the
$\nu$-machine, defined in figure~\ref{machine:nu}, on the set of
problem$_\nu$ to generate a substitution.

The $\delta$-machine in Figure~\ref{machine:delta} computes the final
unifier on three inputs: the substitution resulted from the
$\nu$-machine, $\delta$, and a list of known variables, initialized by
the domain of the substitution.  If no transitions apply, the machine
fails and the unification problem has no unifier.


\begin{lemma}\label{lemma:numachine}
 For all finite input, the $\nu$-machine and the $\delta$-machine
 terminates; for all input, the $\nu$-machine and the $\delta$-machine
 succeeds with the mgu if and only if there exists one.
\end{lemma}
\begin{proof}
    By structural induction on the transitions of the machines.
\end{proof}


Now the question is how to generalize the previous algorithm, that is,
given two arbitrary terms, where a variable may be instantiated by any
term besides atoms, can we re-shape the two terms to create a proper
input to the two machines?


%\begin{wrapfigure}{r}{0.5\textwidth}
\begin{figure}[htbp]
  \begin{minipage}[b]{\textwidth}
  \caption{$\delta$-machine and the pull operation}\label{machine:delta}
  \begin{tabular}{c}
    \fbox{\begin{varwidth}{\textwidth}
        $\dframe{\sigma}{p_\delta}{xs}{\sigma}{p_\delta}$ \\
        $\pframe{\sigma}{xs}{p_\delta}{\sigma}{xs}$
        \end{varwidth}} \\ \\
  \end{tabular}
  \end{minipage}

\begin{minipage}[b]{0.4\textwidth}
  \begin{tabular}{l}
         \infer[\transname{Empty-Q}]{\dframe{\sigma}{\delta}{\epsilon}{\sigma}{\delta}}{%
} \\ \\

         \infer[\transname{Empty-D}]{\dframe{\sigma}{\epsilon}{xs}{\sigma}{\epsilon}}{%
} \\ \\

        \infer[\transname{Step}]{\dframe{\sigma_0}{\delta_0}{X, xs_0}{\sigma_1}{\delta_1}}{%
\dframe{\sigma_0'}{\delta_0 \setminus \delta_0(X)}{xs_1}{\sigma_1}{\delta_1} \\
                                                        \pframe{\sigma_0}{xs_0}{\delta_0(X)}{\sigma_0'}{xs_1}
} \\ \\


\infer[\transname{Empty}]{\pframe{\sigma}{q}{\emptyset}{\sigma}{q}}{%
} \\ \\

\infer[\transname{Name-Name}]{\pframe{\sigma_0}{xs_0}{\eq{\clos{Y_1}{\Phi_1}}{\clos{Y_2}{\Phi_2}}, p}{\sigma_1}{xs_1}}{%
    \pframe{\sigma_0}{xs_0}{p}{\sigma_1}{xs_1} \\
    \aeq{a_1}{\Phi_1}{a_2}{\Phi_2} & \sigma_0(Y_1) = a_1 & \sigma_0(Y_2) = a_2 
} \\ \\

\infer[\transname{Name-Var}]{\pframe{\sigma_0}{xs_0}{\eq{\clos{Y_1}{\Phi_1}}{\clos{Y_2}{\Phi_2}}, p}{\sigma_1}{xs_1}}{%
    \pframe{\{Y_2/a_2\}\cup\sigma_0}{(Y_2, xs_0)}{p}{\sigma_1}{xs_1} &
    \aeq{a_1}{\Phi_1}{a_2}{\Phi_2} & \sigma_0(Y_1) = a_1 & Y_2\notin dom(\sigma_0)} 
\end{tabular}

\end{minipage}
\end{figure}
%\end{wrapfigure}

Finding the common structure, obviously, is merely a first-order
unification problem.  The $\rho$-machine, defined in
figure~\ref{table:rmachine}, adapts the idea of Martelli-Montanari and
reduces an arbitrary nominal unification problem to a set of
problem$_\nu$, a set of problem$_\delta$, and a substitution.  Here we
need to extend the definition of substitution: it is now a partial
finite function from variables to terms.  Also, in the
\transname{var-app} and \transname{var-abs} rules, we need to create
new atoms and new variables.

\begin{wrapfigure}{l}{0.5\textwidth}
%    \begin{figure}
  \caption{$\rho$-machine}\label{table:rmachine}
   \begin{minipage}[b]{\textwidth}
  \begin{tabular}{c}
    \fbox{\begin{varwidth}{\textwidth}
        $\rframe{problem_\nu^*}{problem_\delta^*}{\sigma}{multieqn^*}{problem_\nu^*}{problem_\delta^*}{\sigma}$ \\
        $\sframe{problem_\nu^*}{problem_\delta^*}{\sigma}{multieqn}{problem_\nu^*}{problem_\delta^*}{\sigma}$ \\
        \end{varwidth}} \\ \\
  \end{tabular}
  \end{minipage}

  \begin{minipage}[b]{0.4\textwidth}
  \begin{tabular}{l}
    \infer[\transname{Empty}]{\rframe{p_0}{\delta_0}{\sigma_0}{\emptyset}{p_0}{\delta_0}{\sigma_0}}{%
    } \\  \\

    \infer[\transname{Step}]{\rframe{p_0}{\delta_0}{\sigma_0}{(U, U^*)}{p_1}{\delta_1}{\sigma_1}}{%
    \rframe{p_0'}{\delta_0'}{\sigma_0'}{U^*}{p_1}{\delta_1}{\sigma_1} \\
    \sframe{p_0}{\delta_0}{\sigma_0}{U}{p_0'}{\delta_0'}{\sigma_0'}
    } \\ \\

        \infer[\transname{Name-Name}]{\sframe{p_0}{\delta_0}{\sigma_0}{\pr{\clos{a_1}{\Phi_1}}{\clos{a_2}{\Phi_2}}}{p_1}{\delta_0}{\sigma_0}}{%
    p_1 = \pr{\clos{a_1}{\Phi_1}}{\clos{a_2}{\Phi_2}} \cup p_0
    } \\ \\
        \infer[\transname{Name-Var}]{\sframe{p_0}{\delta_0}{\sigma_0}{\pr{\clos{a_1}{\Phi_1}}{\clos{X_2}{\Phi_2}}}{p_1}{\delta_0}{\sigma_0}}{%
    p_1 = \pr{\clos{a_1}{\Phi_1}}{\clos{X_2}{\Phi_2}} \cup p_0
    } \\ \\
        \infer[\transname{Var-Var}]{\sframe{p_0}{\delta_0}{\sigma_0}{\pr{\clos{a_1}{\Phi_1}}{\clos{X_2}{\Phi_2}}}{p_0}{\delta_1}{\sigma_0}}{%
    \delta_1 = \pr{\clos{X_1}{\Phi_1}}{\clos{X_2}{\Phi_2}} \cup \delta_0
    } \\ \\
        \infer[\transname{App-App}]{\sframe{p_0}{\delta_0}{\sigma_0}{\pr{\clos{\app{l_1}{r_1}}{\Phi_1}}{\clos{\app{l_2}{r_2}}{\Phi_2}}}{p_1}{\delta_1}{\sigma_1}}{%
    \sframe{p_0}{\delta_0}{\sigma_0}{\pr{\clos{l_1}{\Phi_1}}{\clos{l_2}{\Phi_2}}}{p_0'}{\delta_0'}{\sigma_0'} \\
    \sframe{p_0'}{\delta_0'}{\sigma_0'}{\pr{\clos{r_1}{\Phi_1}}{\clos{r_2}{\Phi_2}}}{p_1}{\delta_1}{\sigma_1}
    } \\ \\
            \infer[\transname{Abs-Abs}]{\sframe{p_0}{\delta_0}{\sigma_0}{\pr{\clos{\lambda\,a_1.t_1}{\Phi_1}}{\clos{\lambda\,a_2.t_2}{\Phi_2}}}{p_1}{\delta_1}{\sigma_1}}{%
    \sframe{p_0}{\delta_0}{\sigma_0}{\pr{\clos{t_1}{\Phi_1'}}{\clos{t_2}{\Phi_2'}}}{p_1}{\delta_1}{\sigma_1}\\
    \Phi_1' = (\texttt{ext-scope}\, \Phi_1\, a_1) & \Phi_2' = (\texttt{ext-scope}\, \Phi_2\, a_2)
    } \\ \\
        \infer[\transname{Var-App}]{\sframe{p_0}{\delta_0}{\sigma_0}{\pr{\clos{X_1}{\Phi_1}}{\clos{\app{l_2}{r_2}}{\Phi_2}}}{p_1}{\delta_1}{\sigma_1}}{%
    \sframe{p_0}{\delta_0}{\{X_1/(X_l, X_r)\}\cup\sigma_0'}{\pr{\clos{X_l}{\Phi_1}}{\clos{l_2}{\Phi_2}}}{p_0'}{\delta_0'}{\sigma_0'} \\
    \sframe{p_0'}{\delta_0'}{\sigma_0'}{\pr{\clos{X_r}{\Phi_1}}{\clos{r_2}{\Phi_2}}}{p_1}{\delta_1}{\sigma_1} 
    } \\ \\
        \infer[\transname{Var-Abs}]{\sframe{p_0}{\delta_0}{\sigma_0}{\pr{\clos{X_1}{\Phi_1}}{\clos{\lambda\,a_2.t_2}{\Phi_2}}}{p_1}{\delta_1}{\sigma_1}}{%
    \sframe{p_0}{\delta_0}{\bd{X_1}{\lambda\,a_1.X_t}\cup\sigma_0'}{\pr{\clos{X_t}{\Phi_1'}}{\clos{t_2}{\Phi_2'}}}{p_1}{\delta_1}{\sigma_1} \\
    \Phi_1' = (\texttt{ext-scope}\, \Phi_1\, a_1) & \Phi_2' = (\texttt{ext-scope}\, \Phi_2\, a_2) 
    } \\ \\



    % \transname{empty} & $\rframe{p_0}{\delta_0}{\sigma_0}{\emptyset}{p_0}{\delta_0}{\sigma_0}$ \\
    % \transname{step} & $\rframe{p_0}{\delta_0}{\sigma_0}{(U,\,U^*)}{p_1}{\delta_1}{\sigma_1}$ \\
    %                       & where $\rframe{p_0'}{\delta_0'}{\sigma_0'}{U*}{p_1}{\delta_1}{\sigma_1}$ \\
    %                       & \hspace{8mm} $\cframe{p_0}{\delta_0}{\sigma_0}{U}{p_0'}{\delta_0'}{\sigma_0'}$ \\
    %                   & \\
    % \transname{name-name} & $\cframe{p_0}{\delta_0}{\sigma_0}{\pr{\clos{a_1}{\Phi_1}}{\clos{a_2}{\Phi_2}}}{p_0'}{\delta_0}{\sigma_0}$ \\
   %                        & where $p_0' = {\pr{\clos{a_1}{\Phi_1}}{\clos{a_2}{\Phi_2}}} \cup p_0$  \\
   %  \transname{name-var} & $\cframe{p_0}{\delta_0}{\sigma_0}{\pr{\clos{a_1}{\Phi_1}}{\clos{X_2}{\Phi_2}}}{p_0'}{\delta_0}{\sigma_0}$ \\
   %                        & where $p_0' = {\pr{\clos{a_1}{\Phi_1}}{\clos{X_2}{\Phi_2}}} \cup p_0$  \\
   %  \transname{var-var} & $\cframe{p_0}{\delta_0}{\sigma_0}{\pr{\clos{X_1}{\Phi_1}}{\clos{X_2}{\Phi_2}}}{p_0}{\delta_0'}{\sigma_0}$ \\
   %                        & where $p_0' = {\pr{\clos{X_1}{\Phi_1}}{\clos{X_2}{\Phi_2}}} \cup p_0$  \\
   %  \transname{app-app} & $\cframe{p_0}{\delta_0}{\sigma_0}{\pr{\clos{\app{l_1}{r_1}}{\Phi_1}}{\clos{\app{l_2}{r_2}}{\Phi_2}}}{p_1}{\delta_1}{\sigma_1}$ \\ 
   %  & where $\cframe{p_0}{\delta_0}{\sigma_0}{\pr{\clos{l_1}{\Phi_1}}{\clos{l_2}{\Phi_2}}}{p_0'}{\delta_0'}{\sigma_0'}$ \\
   %  & \hspace{8mm} $\cframe{p_0'}{\delta_0'}{\sigma_0'}{\pr{\clos{r_1}{\Phi_1}}{\clos{r_2}{\Phi_2}}}{p_1}{\delta_1}{\sigma_1}$ \\
   %  \transname{abs-abs} & $\cframe{p_0}{\delta_0}{\sigma_0}{\pr{\clos{\lambda\,a_1.t_1}{\Phi_1}}{\clos{\lambda\,a_2.t_2}{\Phi_2}}}{p_1}{\delta_1}{\sigma_1}$ \\ 
   %  & where $\cframe{p_0}{\delta_0}{\sigma_0}{\pr{\clos{t_1}{\Phi_1'}}{\clos{t_2}{\Phi_2'}}}{p_1}{\delta_1}{\sigma_1}$ \\
   %  & \hspace{8mm} $\Phi_1' = (\texttt{ext-scope}\, \Phi_1\, a_1)$ \\
   %  & \hspace{8mm} $\Phi_2' = (\texttt{ext-scope}\, \Phi_2\, a_2)$ \\
   %  \transname{var-app} & $\cframe{p_0}{\delta_0}{\sigma_0}{\pr{\clos{X_1}{\Phi_1}}{\clos{\app{l_2}{r_2}}{\Phi_2}}}{p_1}{\delta_1}{\sigma_1}$ \\ 
   %  & where $\cframe{p_0}{\delta_0}{\sigma_0'}{\pr{\clos{X_l}{\Phi_1}}{\clos{l_2}{\Phi_2}}}{p_0'}{\delta_0'}{\sigma_0''}$ \\
   %  & \hspace{8mm} $\cframe{p_0'}{\delta_0'}{\sigma_0''}{\pr{\clos{X_r}{\Phi_1}}{\clos{r_2}{\Phi_2}}}{p_1}{\delta_1}{\sigma_1}$ \\
   %  & \hspace{8mm} $\sigma_0' = \{\bd{X_1}{\app{X_l}{X_r}}\} \cup \sigma_0$ \\
   % \transname{var-abs} & $\cframe{p_0}{\delta_0}{\sigma_0}{\pr{\clos{X_1}{\Phi_1}}{\clos{\lambda\,a_2.t_2}{\Phi_2}}}{p_1}{\delta_1}{\sigma_1}$ \\ 
   %  & where $\cframe{p_0}{\delta_0}{\sigma_0'}{\pr{\clos{X_t}{\Phi_1'}}{\clos{t_2}{\Phi_2'}}}{p_1}{\delta_1}{\sigma_1}$ \\
   %  & \hspace{8mm} $\Phi_1' = (\texttt{ext-scope}\, \Phi_1\, a_1)$ \\
   %  & \hspace{8mm} $\Phi_2' = (\texttt{ext-scope}\, \Phi_2\, a_2)$ \\
   %  & \hspace{8mm} $\sigma_0' = \{\bd{X_1}{\lambda\,a_1.X_t}\} \cup \sigma_0$ \\

  \end{tabular}
  \end{minipage}
%\end{figure}
\end{wrapfigure}

\section{A note on time complexity}
    \label{efficiency}

In the previous sections, we represent scopes by lists for simplicity,
but lists are inefficient for variable lookup.  To have better time
complexity, we represent a scope with a counter and two persistent
hashtables, also known as hamt~\citep{bagwell_ideal_2001}.  One
hashtable maps from atoms to numbers, the other maps from numbers to
atoms, and the counter is used to track the de Bruijn number.  When we
extend a scope with a atom, we extend the two hashtables with the
corresponding maps and add one to the counter.  An persistent
hashtable, in practice, has constant time for update and lookup,
thought the worst case scenario could be $O(log(n))$.  Thus,
\texttt{ext-scope}, \texttt{idx$\rightarrow$atom}, and
\texttt{atom$\rightarrow$idx} are all logarithmic time.  In addition,
using persistent structures avoids copying the entire data-structure
when branching, in particular, during the \transname{app-app} rule of
the $\rho$-machine.

We implement $\delta$ with a hashtable that maps from a variable to
the list that contains its closure-equations.  Doing so doubles the
space consumption, i.e, the equation $\aeq{X}{\Phi_1}{Y}{\Phi_2}$
exists in both $X$'s entry and $Y$'s entry, but improves the time
efficiency.

Given the above optimizations, the $\nu$-machine and the
$\delta$-machine are both $O(n \times log(n))$ at the worst case,
where $n$ is the number of atom occurrences and variable occurrences.
The algorithm of Martelli-Montanari is $O(n \times G(n))$, when
representing sets with UNION-FIND~\citep{tarjan_efficiency_1975}, where
$n$ is the number of variable occurrences in the original terms.  The
$\rho$-machine is similar except that two new factors are involved:
the update operation of an hamt and the generation of atoms and
variables.  We consider the former one to have $O(log(n))$ complexity,
and we implement atom and variable creation with state monads in
constant time.  Thus reducing an arbitrary unification problem to the
input of the $\nu$ and $\delta$ machines is $O(n \times log(n) \times
G(n))$.

    \clearpage

\bibliography{main}

\end{document}
%%% Local Variables:
%%% mode: latex
%%% TeX-master: t
%%% End:
